% Options for packages loaded elsewhere
\PassOptionsToPackage{unicode}{hyperref}
\PassOptionsToPackage{hyphens}{url}
\PassOptionsToPackage{dvipsnames,svgnames,x11names}{xcolor}
%
\documentclass[
  letterpaper,
  DIV=11,
  numbers=noendperiod]{scrartcl}

\usepackage{amsmath,amssymb}
\usepackage{lmodern}
\usepackage{iftex}
\ifPDFTeX
  \usepackage[T1]{fontenc}
  \usepackage[utf8]{inputenc}
  \usepackage{textcomp} % provide euro and other symbols
\else % if luatex or xetex
  \usepackage{unicode-math}
  \defaultfontfeatures{Scale=MatchLowercase}
  \defaultfontfeatures[\rmfamily]{Ligatures=TeX,Scale=1}
\fi
% Use upquote if available, for straight quotes in verbatim environments
\IfFileExists{upquote.sty}{\usepackage{upquote}}{}
\IfFileExists{microtype.sty}{% use microtype if available
  \usepackage[]{microtype}
  \UseMicrotypeSet[protrusion]{basicmath} % disable protrusion for tt fonts
}{}
\makeatletter
\@ifundefined{KOMAClassName}{% if non-KOMA class
  \IfFileExists{parskip.sty}{%
    \usepackage{parskip}
  }{% else
    \setlength{\parindent}{0pt}
    \setlength{\parskip}{6pt plus 2pt minus 1pt}}
}{% if KOMA class
  \KOMAoptions{parskip=half}}
\makeatother
\usepackage{xcolor}
\setlength{\emergencystretch}{3em} % prevent overfull lines
\setcounter{secnumdepth}{-\maxdimen} % remove section numbering
% Make \paragraph and \subparagraph free-standing
\ifx\paragraph\undefined\else
  \let\oldparagraph\paragraph
  \renewcommand{\paragraph}[1]{\oldparagraph{#1}\mbox{}}
\fi
\ifx\subparagraph\undefined\else
  \let\oldsubparagraph\subparagraph
  \renewcommand{\subparagraph}[1]{\oldsubparagraph{#1}\mbox{}}
\fi


\providecommand{\tightlist}{%
  \setlength{\itemsep}{0pt}\setlength{\parskip}{0pt}}\usepackage{longtable,booktabs,array}
\usepackage{calc} % for calculating minipage widths
% Correct order of tables after \paragraph or \subparagraph
\usepackage{etoolbox}
\makeatletter
\patchcmd\longtable{\par}{\if@noskipsec\mbox{}\fi\par}{}{}
\makeatother
% Allow footnotes in longtable head/foot
\IfFileExists{footnotehyper.sty}{\usepackage{footnotehyper}}{\usepackage{footnote}}
\makesavenoteenv{longtable}
\usepackage{graphicx}
\makeatletter
\def\maxwidth{\ifdim\Gin@nat@width>\linewidth\linewidth\else\Gin@nat@width\fi}
\def\maxheight{\ifdim\Gin@nat@height>\textheight\textheight\else\Gin@nat@height\fi}
\makeatother
% Scale images if necessary, so that they will not overflow the page
% margins by default, and it is still possible to overwrite the defaults
% using explicit options in \includegraphics[width, height, ...]{}
\setkeys{Gin}{width=\maxwidth,height=\maxheight,keepaspectratio}
% Set default figure placement to htbp
\makeatletter
\def\fps@figure{htbp}
\makeatother

\KOMAoption{captions}{tableheading}
\makeatletter
\makeatother
\makeatletter
\makeatother
\makeatletter
\@ifpackageloaded{caption}{}{\usepackage{caption}}
\AtBeginDocument{%
\ifdefined\contentsname
  \renewcommand*\contentsname{Table of contents}
\else
  \newcommand\contentsname{Table of contents}
\fi
\ifdefined\listfigurename
  \renewcommand*\listfigurename{List of Figures}
\else
  \newcommand\listfigurename{List of Figures}
\fi
\ifdefined\listtablename
  \renewcommand*\listtablename{List of Tables}
\else
  \newcommand\listtablename{List of Tables}
\fi
\ifdefined\figurename
  \renewcommand*\figurename{Figure}
\else
  \newcommand\figurename{Figure}
\fi
\ifdefined\tablename
  \renewcommand*\tablename{Table}
\else
  \newcommand\tablename{Table}
\fi
}
\@ifpackageloaded{float}{}{\usepackage{float}}
\floatstyle{ruled}
\@ifundefined{c@chapter}{\newfloat{codelisting}{h}{lop}}{\newfloat{codelisting}{h}{lop}[chapter]}
\floatname{codelisting}{Listing}
\newcommand*\listoflistings{\listof{codelisting}{List of Listings}}
\makeatother
\makeatletter
\@ifpackageloaded{caption}{}{\usepackage{caption}}
\@ifpackageloaded{subcaption}{}{\usepackage{subcaption}}
\makeatother
\makeatletter
\@ifpackageloaded{tcolorbox}{}{\usepackage[many]{tcolorbox}}
\makeatother
\makeatletter
\@ifundefined{shadecolor}{\definecolor{shadecolor}{rgb}{.97, .97, .97}}
\makeatother
\makeatletter
\makeatother
\ifLuaTeX
  \usepackage{selnolig}  % disable illegal ligatures
\fi
\IfFileExists{bookmark.sty}{\usepackage{bookmark}}{\usepackage{hyperref}}
\IfFileExists{xurl.sty}{\usepackage{xurl}}{} % add URL line breaks if available
\urlstyle{same} % disable monospaced font for URLs
\hypersetup{
  pdftitle={Mappeoppgave 1 - Økonomi, statistikk og programmering},
  pdfauthor={Kandidatnummer 6, SOK-2009, Høst 2023},
  colorlinks=true,
  linkcolor={blue},
  filecolor={Maroon},
  citecolor={Blue},
  urlcolor={Blue},
  pdfcreator={LaTeX via pandoc}}

\title{Mappeoppgave 1 - Økonomi, statistikk og programmering}
\usepackage{etoolbox}
\makeatletter
\providecommand{\subtitle}[1]{% add subtitle to \maketitle
  \apptocmd{\@title}{\par {\large #1 \par}}{}{}
}
\makeatother
\subtitle{Fakultet for biovitenskap, fiskeri og økonomi.}
\author{Kandidatnummer 6, SOK-2009, Høst 2023}
\date{05-10-2023}

\begin{document}
\maketitle
\ifdefined\Shaded\renewenvironment{Shaded}{\begin{tcolorbox}[boxrule=0pt, frame hidden, sharp corners, borderline west={3pt}{0pt}{shadecolor}, interior hidden, breakable, enhanced]}{\end{tcolorbox}}\fi

\renewcommand*\contentsname{Innholdsliste}
{
\hypersetup{linkcolor=}
\setcounter{tocdepth}{3}
\tableofcontents
}
\hypertarget{oppgave-5}{%
\subsection{\texorpdfstring{Oppgave
5\textsuperscript{*}}{Oppgave 5*}}\label{oppgave-5}}

Anta at du har to sett med to seksidede terninger. Noen andre kaster terningene og skriver ned summen av prikker fra begge terningene; (T_{12}) og (T_{22}). Du får ikke se terningene men får vite at summen fra sett en er større en fra sett to, dvs. (T_{12} > T_{22}). La oss si at du tar ut to terninger, en fra sett en (t_{11}) og en fra sett to (t_{22}).

\begin{enumerate}
    \item Er forventningverdien lik for disse to terningene? Forklar svaret ditt.
    \item Tegn en graf med mulige utfall for \(T_1\) og \(T_2\). Hvilke utfall er mulige og hvilke er ikke mulige gitt \(T_{12} > T_{22}\)? Er det uniform sannsynlighet for utfallene?
    \item Bruk R og regn ut forventningsverdien til de to terningene.
\end{enumerate}

\begin{enumerate}
    \item Formelen for trekning av \(X\) mulige \(r\) ganger med tilbakelegging er \(X^r\). For fire seksidede terninger altså \(X = 6, r = 4\). Sjekk også \url{https://tma4245.math.ntnu.no/hendelser-og-sannsynlighet/uniform-sannsynlighetsmodell/kombinatorikk-ordnet-utvalg-trekning-med-tilbakelegging/} eller \url{https://www.hackmath.net/en/calculator/combinations-and-permutations}.
    \item Lag \(36 \times 36\) vektorer med et tall for hver terning, det vil si \([(1,1)(1,1)], [(1,1)(1,2)],[(1,1)(1,3)],\ldots, [(6,6)(6,6)]\). Dette er utfallsrommet for to sett med to terninger. Du må muligens bruke en spesial kommando for å få til dette datasettet.
    \item Fjern alle vektorer der terningsett en er større enn terningsett to, dvs. der \([(1, 1) < (1, 2)]\).
    \item Du skal nå sitte igjen med litt mindre enn halvparten av de \([36 \times 36 ]\) utfallene. (Disse oppfyller at summen av terningkast på to er større enn en \(T_{12} > T_{22}\).)
    \item Kalkuler forventningen til den første av terningen. Alle utfall er like sannsynlige, så dette er lett.
\end{enumerate}

Sjekk kode på:
\url{https://stackoverflow.com/questions/45878448/creating-sample-space-in-r}



\end{document}
